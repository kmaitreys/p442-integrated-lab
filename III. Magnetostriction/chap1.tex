\chapter{\label{intro}Introduction}

In the early 1800s, the study of electricity and magnetism was a burgeoning field, full of exciting discoveries and new insights. Hailing from Lancashire, English physicist James Joule, namesake for the SI unit of energy and heat, was studying the ways in which magnetic fields interacted with matter. He had already made a name for himself in the scientific community for his work on the relationship between heat and energy, but he was eager to explore the ways in which magnets could affect the behavior of materials.

In 1842, Joule made a groundbreaking discovery that would change the course of his research and contribute to the development of a new field of study: magnetostriction. While experimenting with a sample of iron that had been placed in a magnetic field, Joule noticed something peculiar. As he varied the strength of the field, he observed that the iron sample actually changed in size. When the field was turned off, the sample returned to its original dimensions.

Joule was fascinated by this behavior and began to explore it in more detail. He found that other materials, including nickel and cobalt, exhibited similar properties under the influence of a magnetic field. He called this phenomenon ``magnetostriction,'' from the Greek words for ``magnet'' and ``to twist.''

Joule's discovery opened up a whole new area of research, as scientists sought to understand the underlying mechanisms behind magnetostriction and explore its potential applications. Over time, they discovered that magnetostriction was closely related to other magnetic effects, such as the magneto-optic effect and the giant magnetoresistance effect. Today, magnetostriction continues to be an active area of research, with applications in fields ranging from materials science to electrical engineering as it allows storing electrical energy in form of mechanical energy at scale.

In this term report, we used a Michelson interferometer setup paired with a He-Ne laser to study this phenomena.
\section{Objectives}

The following objectives were formulated and achieved for this experiment:

\begin{enumerate}
	\item  To study the phenomena of magnetostriction in Nickel, Iron and Copper rods.
	\item To measure longitudinal strain with the applied varying magnetic field.
\end{enumerate}


\setcounter{equation}{0}
\setcounter{table}{0}
\setcounter{figure}{0}
%\baselineskip 24pt


    



