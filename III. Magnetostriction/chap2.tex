\chapter{\label{method}Theory}
In this chapter, the theoretical background will be detailed before we discuss the particulars of the experiment.
\section{Magnetostriction}
The internal structure of ferromagnetic materials is composed of domains, each of which is a region of uniform magnetization. The domain boundaries move and the domains rotate when a magnetic field is applied; both of these effects change the material's dimensions. Because it requires more energy to magnetize a crystalline material in one direction than in another, magnetocrystalline anisotropy\footnote{A ferromagnetic material is said to have magnetocrystalline anisotropy if it takes more energy to magnetize it in certain directions than in others.} is the reason that altering a material's magnetic domains causes a change in the material's dimensions. The material will tend to rearrange its structure so that an easy axis is aligned with the field to minimize the free energy of the system if a magnetic field is applied to it at an angle to an easy axis of magnetization\footnote{Easy axis refers to the energetically favorable direction of the spontaneous magnetization in a ferromagnetic material.}. This differential induction of direction leads to strain in the material.

In fact, the reciprocal effect is also observed in the form of Villari effect, which is defined as the change in the magnetic susceptibility (which quantifies the response to an applied field) of a magnetic material under some form of mechanical stress. The Matteucci and Wiedemann effects are also two related phenomena where mechanical transformation of ferromagnetic materials leads to changes in their magentic properties.

The changes in dimensions of the material are always in the direction of magnetization, and it they can be positive or negative depending on if the material is contracting or elongating. The distortions are usually of the order $10^{-8}$ to $10^{-4} \si{\meter}$.
\section{LASER}

\section{Michelson Interferometer}
\section{Photo-diode Detector}

\setcounter{equation}{0}
\setcounter{table}{0}
\setcounter{figure}{0}
%\baselineskip 24pt


    



