\chapter{\label{results}Results and Discussion}
\begin{enumerate}
	\item An interference pattern was observed.
	\item Magnetostriction coefficients were measured in a magnetic optical bench with a standard arm length for Fe, Ni, and Cu. The results showed that Fe had a coefficient of $(1.38 \pm 0.58) \times 10^{-6}$, Ni had a coefficient of $(-5.53 \pm 3.1) \times 10^{-6}$, and Cu did not show any magnetostriction.
	\item Magnetostriction coefficients were measured in a mechanical optical bench with an extended arm length for Fe, Ni, and Cu. The results showed that Fe had a coefficient of $(1.84 \pm 1.1) \times 10^{-6}$, Ni had a coefficient of $(-8.57 \pm 4.9) \times 10^{-6}$, and Cu did not show any magnetostriction.
	\item These results were in agreement with the literature values.
	\item Because Cu is not ferromagnetic, it does not have magnetic domains in the bulk material, and therefore, it was not expected to exhibit any magnetostriction, which was consistent with the observations.

	\item The op-amp circuit connected to the photodetector had some loose connections, causing the data to be unreliable. The source of this error was identified, and the loose wires were resoldered to their proper places to rectify the issue.
	\item The optical bench experienced vibrations, leading to the disappearance of fringes. To solve this problem, a vibration damping table was utilized to minimize the impact of the vibrations.
	\item The laser beam was not aligned with the plane of the optical bench, resulting in the interference not being observed. To fix this issue, a spirit level was used to adjust the beam's position and bring it back into alignment with the plane.
	\item The LASER source has a narrow divergence in its beam over a long distance, but a diverging lens was used in the experiment to expand the beam into a circular pattern for ease in counting rings. However, this also meant that the beam could not be used to produce interference patterns for longer arm lengths since the beam would diverge too much to provide relevant data.
	
	\item To overcome this limitation, the lens was removed in the experiment with the longer arm length, and the device was positioned to take readings of standard interference fringes instead of circular fringes.
	
	\item Although a change in arm length should not theoretically affect the observation of fringes, a slight variation in the results was observed. This could be attributed to the fact that the standard interferometer pattern gets less intense after the first few bright fringes, unlike circular fringes, which remain equally bright over a larger region.
	
	\item In future experiments, a Mach-Zehnder interferometer could be used to increase the instrument's accuracy.
\end{enumerate}
\setcounter{equation}{0}
\setcounter{table}{0}
\setcounter{figure}{0}
%\baselineskip 24pt


    



