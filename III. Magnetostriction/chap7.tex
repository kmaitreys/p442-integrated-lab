\chapter{\label{summary}Summary and Conclusions}
\begin{enumerate}
	\item To summarize, the text describes an experiment where interference patterns were observed, and magnetostriction coefficients were measured for Fe, Ni, and Cu using both a magnetic and mechanical optical bench. The results show that Cu did not exhibit any magnetostriction, which was expected due to its non-ferromagnetic nature.
	\item During the experiment, the reliability of the data was compromised due to loose connections in the op-amp circuit. Vibrations in the optical bench also led to issues with the fringes, which were resolved by utilizing a vibration damping table. Additionally, the laser beam was not in the right position, and the interference was not observed, but this was rectified by adjusting the beam's position using a spirit level.
	\item The experiment used a LASER source with a narrow beam divergence, which was expanded into a circular pattern using a diverging lens for ease in counting rings. However, this limited the experiment's ability to produce interference patterns for longer arm lengths. To overcome this, the lens was removed for the longer arm experiment, and standard interference fringes were observed instead of circular fringes. Despite a slight variation in the results, a Mach-Zehnder interferometer could be used in future experiments to increase the instrument's accuracy.
\end{enumerate}

\setcounter{equation}{0}
\setcounter{table}{0}
\setcounter{figure}{0}
%\baselineskip 24pt


    



