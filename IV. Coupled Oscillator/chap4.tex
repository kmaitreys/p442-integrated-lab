\chapter{\label{method}Data Analysis}
The data analysis was performed using Python's \texttt{SciPy} library. We recorded the data for 20 combinations of masses attached to the oscillators and then for each file, the fitted parameters were found. Corresponding to each fit parameter, we found the the Rabi model parameters of $ Delta $ (detuning) and $ \Omega $ (Rabi frequency). Then these parameters of the Rabi model were themselves fit according to their equations, relating them with geometric phase.

For error analysis, we used Python's  \texttt{uncertainties} package, which preserved the significant digits when propagating errors. The complete code is on my \href{https://github.com/peakcipher/p442-integrated-lab/tree/master/IV.%20Coupled%20Oscillator}{GitHub}. Some snippets are also attached with this report.

In this section, we present our plots. In figure \ref{fig:first}, we present the fitted curve for one of the data file (representing one mass combination for the coupled oscillator). The figure \ref{fig:first_sq} represents the squares of the normalised amplitudes (basically the peaks of the `carrier' wave).

\begin{figure}[H]
	\centering
	\includegraphics[scale=0.4]{01.png}
	\caption{The amplitude versus current data obtained for Fe rod on the magnetic bench}
	\label{fig:first}
\end{figure}

\begin{figure}[H]
	\centering
	\includegraphics[scale=0.4]{01_squared.png}
	\caption{The amplitude versus current data obtained for Fe rod on the magnetic bench}
	\label{fig:first_sq}
\end{figure}

Similarly the other plots were obtained and that gave us the fitting parameters. In total we used 19 data files, each of which gave the Rabi model parameters we needed to study the dependence of geometric phase with detuning. 

\section{Calculation of Rabi model parameters and the Geometric Phase}
Here we present the plots where we calculated the Rabi model parameters and the geometric phase and fitted thier functions.

\begin{figure}[H]
	\centering
	\includegraphics[scale=0.4]{20_freq_det.png}
	\caption{The amplitude versus current data obtained for Fe rod on the magnetic bench}
	\label{fig:first_sq}
\end{figure}

\begin{figure}[H]
	\centering
	\includegraphics[scale=0.4]{Geometric_phase.png}
	\caption{The amplitude versus current data obtained for Fe rod on the magnetic bench}
	\label{fig:first_sq}
\end{figure}

\begin{figure}[H]
	\centering
	\includegraphics[scale=0.4]{Zoomed_in_geometric_phase.png}
	\caption{The amplitude versus current data obtained for Fe rod on the magnetic bench}
	\label{fig:first_sq}
\end{figure}
\setcounter{equation}{0}
\setcounter{table}{0}
\setcounter{figure}{0}
%\baselineskip 24pt
