\chapter{\label{intro}Introduction}

The study of classical and quantum mechanics is one of the most fascinating and essential topics in physics. The two disciplines have been the cornerstone of modern physics and have revolutionized our understanding of the natural world. Classical mechanics is the branch of physics that describes the behavior of macroscopic objects, while quantum mechanics deals with the behavior of subatomic particles. Throughout history, many great minds, such as Isaac Newton and Albert Einstein, have made significant contributions to classical mechanics. Similarly, quantum mechanics has been shaped by the works of notable physicists such as Max Planck, Erwin Schrödinger, and Werner Heisenberg. The classical and quantum analogies have played a pivotal role in bridging the gap between the two fields and have led to many exciting discoveries in the world of physics.

One of the interesting areas where classical and quantum analogies are commonly used is in the study of cyclical dynamical systems. In classical mechanics, a simple example of a cyclical system is a set of coupled oscillators. These oscillators exhibit a periodic behavior where their motions are linked and synchronized. In the realm of quantum mechanics, the equivalent system is a two-level quantum system, also known as a qubit. Qubits are the basic building blocks of quantum computing and exhibit similar periodic behaviors as coupled oscillators. The study of these systems has led to a better understanding of the behavior of complex systems and their underlying principles. Moreover, the classical-quantum analogy has allowed researchers to gain insights into quantum systems by studying classical analogs and vice versa. This has been an essential tool in the development of quantum technologies such as quantum computing, cryptography, and communication.

\section{Objectives}

The following objectives were formulated and achieved for this experiment:

\begin{enumerate}
	\item To observe and analyse various fit parameters related to the dynamics of the coupled oscillator
	\item To study the variation of geometric phase of the coupled oscillator
	\item To analyse the correlation matrix to validate the quality of fit parameters
	\item To study the effects of a mechanical damping agent on the coupled oscillator.
\end{enumerate}


\setcounter{equation}{0}
\setcounter{table}{0}
\setcounter{figure}{0}
%\baselineskip 24pt


    



